\documentclass[11pt]{article}

\usepackage{finalreport} % uncomment this line for a final report
% \usepackage{progressreport} % uncomment this line for a progress report
\usepackage[T1]{fontenc}
\usepackage{hyperref}
\usepackage{url}
\usepackage{booktabs}
\usepackage{amsfonts}
\usepackage{nicefrac}
\usepackage{microtype}
\usepackage{amsmath}
\usepackage{cleveref}
\usepackage{lipsum}
\usepackage{graphicx}
\usepackage{natbib}
\usepackage{doi}
\usepackage{algpseudocode}
\usepackage{algorithm}
\usepackage{algpseudocode}
\usepackage{listings}
\usepackage{xcolor}
\usepackage{graphicx}
\usepackage{fontspec}
\setmainfont{Times New Roman}
\newfontfamily\chinesefont{Noto Serif TC}
\usepackage{setspace}
\usepackage{nomencl}
\makenomenclature

\hypersetup{
  pdftitle={\@title},
  pdfauthor={\@author},
}

% metadata for the  project
\courseproject{The project course title here}
\title{Your title here}
\author{Your name here}
\uid{Your UID here}
\supervisor{Your supervisor's name here}
\secondexaminer{Your second examiner's name here}


\begin{document}

\maketitle
\frontmatter
\input{sections/abstract.tex}
\begin{acknowledgment}
  \lipsum[1]
\end{acknowledgment}

\makefrontmatter
\renewcommand{\nomname}{Abbreviations}
\addcontentsline{toc}{section}{Abbreviations}
\printnomenclature

% list your abbreviation here
\nomenclature{LHS}{Left Hand Side}
\nomenclature{RHP}{Right Hand Side}


\mainmatter
\section{What is \LaTeX{}?}
\LaTeX{} is a typesetting system that allows you to focus on the content of your document rather than its appearance.

\section{Basic Elements of a \LaTeX{} Document}
Here is a list of some of the basic elements you'll often see in \LaTeX{} documents:

\begin{itemize}
  \item Document class declaration: This is where you specify the type of document (like article, book, etc.).
  \item Packages: \LaTeX{} packages add extra functionality to \LaTeX{}.
  \item Body: The body of the document is enclosed inside the \textbackslash begin\{document\} and \textbackslash end\{document\} commands.
\end{itemize}

\section{Creating Sections and Subsections}
Use \textbackslash section\{Section Heading\} to create a section. For subsections, use \textbackslash subsection\{Subsection Heading\}.

\section{Adding Images}
To include images in your document, you need to use the `graphicx` package. Then you can use the \textbackslash includegraphics command to include your image. For example,

\textbackslash begin\{figure\}[h]
\textbackslash includegraphics[width=\textbackslash textwidth]{file.jpg}
\textbackslash caption\{Image Caption\}
\textbackslash end\{figure\}

\section{Creating Equations}

There are multiple ways to add equations in \LaTeX{}. A simple way is to use the `amsmath` package and use \textbackslash begin\{equation\} ... \textbackslash end\{equation\} environment. For example,

\textbackslash begin\{equation\}
% E = mc^2
\textbackslash end\{equation\}

\begin{equation}
  E = mc^2
\end{equation}

\section{Creating Tables in \LaTeX{}}
Tables in \LaTeX{} can be a bit more complex, but here's a simple example:

\begin{table}[h]
  \centering
  \begin{tabular}{|l|l|}
    \hline
    \textbf{Header 1} & \textbf{Header 2} \\
    \hline
    Item 1            & Item 2            \\
    Item 3            & Item 4            \\
    \hline
  \end{tabular}
  \caption{A Simple Table}
  \label{table:1}
\end{table}

\section{Referencing Figures, Tables, and Sections}
As shown in \Cref{table:1} and \Cref{fig:1}, \LaTeX{} allows cross-referencing figures, tables, and sections. Make sure to add a label after the caption like this:

\begin{verbatim}
 \caption{A Simple Table}
 \label{table:1}
\end{verbatim}

Here is a figure for demonstration:

\begin{figure}[h]
  \centering
  \includegraphics[width=0.4\textwidth]{assets/placeholder.jpg}
  \caption{A sample image}
  \label{fig:1}
\end{figure}

Now you can refer back to these labels in your text using \textbackslash Cref.

\section{Creating a Bibliography}
A document often has a bibliography. Though there are a few methods to add one, we will use BibTeX in this example. Suppose we have a BibTeX file called references.bib with an entry:

\begin{verbatim}
@book{latexcompanion,
author    = "Michel Goossens and Frank Mittelbach and Alexander Samarin",
title     = "The \LaTeX\ Companion",
year      = "1993",
publisher = "Addison-Wesley",
address   = "Reading, Massachusetts"
}
\end{verbatim}

To cite this source in your document, use \textbackslash cite\{key\}, where key is the BibTeX entry key. Like so: \cite{latexcompanion}. Don't forget to reference your `.bib` file at the end of your document with:

\begin{verbatim}
\bibliography{references}
\end{verbatim}

And choose your style with \textbackslash bibliographystyle\{style\}. e.g.

\begin{verbatim}
\bibliographystyle{plain}
\end{verbatim}

\section{Conclusion}
Please replace the placeholder content with your actual content. This is just a starting point and \LaTeX{} offers a far more rich set of features.

\backmatter
\makereferences
\appendix
\section{Appendix A}
\lipsum[1]


\end{document}
